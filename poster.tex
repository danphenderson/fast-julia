%%%%%%%%%%%%%%%%%%%%%%%%%%%%%%%%%%%%%%%%%
% a0poster Landscape Poster
% LaTeX Template
% Version 1.0 (22/06/13)
%
% The a0poster class was created by:
% Gerlinde Kettl and Matthias Weiser (tex@kettl.de)
% 
% This template has been downloaded from:
% http://www.LaTeXTemplates.com
%
% License:
% CC BY-NC-SA 3.0 (http://creativecommons.org/licenses/by-nc-sa/3.0/)
%
%%%%%%%%%%%%%%%%%%%%%%%%%%%%%%%%%%%%%%%%%

%----------------------------------------------------------------------------------------
%	PACKAGES AND OTHER DOCUMENT CONFIGURATIONS
%----------------------------------------------------------------------------------------

\documentclass[a0,landscape]{a0poster}

\usepackage{multicol} % This is so we can have multiple columns of text side-by-side
\columnsep=100pt % This is the amount of white space between the columns in the poster
\columnseprule=3pt % This is the thickness of the black line between the columns in the poster

\usepackage[svgnames]{xcolor} % Specify colors by their 'svgnames', for a full list of all colors available see here: http://www.latextemplates.com/svgnames-colors

\usepackage{times} % Use the times font
%\usepackage{palatino} % Uncomment to use the Palatino font

\usepackage{graphicx} % Required for including images
\graphicspath{{figures/}} % Location of the graphics files
\usepackage{booktabs} % Top and bottom rules for table
\usepackage[font=small,labelfont=bf]{caption} % Required for specifying captions to tables and figures
\usepackage{amsfonts, amsmath, amsthm, amssymb} % For math fonts, symbols and environments

\definecolor{PMS310C}{RGB}{62,192,197} % official university color 2025/07/01

\begin{document}

%----------------------------------------------------------------------------------------
%	POSTER HEADER 
%----------------------------------------------------------------------------------------

% The header is divided into three boxes:
% The first is 55% wide and houses the title, subtitle, names and university/organization
% The second is 25% wide and houses contact information
% The third is 19% wide and houses a logo for your university/organization or a photo of you
% The widths of these boxes can be easily edited to accommodate your content as you see fit

\begin{minipage}[b]{0.55\linewidth}
  \raggedright
  \veryHuge \color{PMS310C} \textbf{POC to Production} \color{Black}\\[0.4cm]
  \Huge\textit{Fast Julia Dynamical Systems}\\[0.8cm]
\end{minipage}
\hfill
\begin{minipage}[b]{0.23\linewidth}
  \raggedright
  \Large \textbf{Contact Information:}\\[0.25cm]
  Daniel Henderson\\
  Michigan Technological University\\
  Department of Mathematical Sciences\\
  Houghton, MI\\
  Email: \texttt{dphender@mtu.edu}
\end{minipage}%
\hfill
\begin{minipage}[b]{0.18\linewidth}
  \raggedleft
  \vspace{0.4cm}
  \includegraphics[width=0.9\linewidth]{figures/logo.png}\\[0.6cm]
\end{minipage}

\vspace{1cm}
\hrulefill

%----------------------------------------------------------------------------------------

\begin{multicols}{3} % This is how many columns your poster will be broken into, 
% a poster with many figures may benefit from less columns whereas a text-heavy poster benefits from more

%----------------------------------------------------------------------------------------
%	ABSTRACT
%----------------------------------------------------------------------------------------

\begin{abstract}
  \noindent Comparison of niave to production implementations for solve small system
  of ODEs $(< 100)$ (why less than 100?)
\end{abstract}

%----------------------------------------------------------------------------------------

\section*{Introduction}


\section*{Results/Analysis}

\section*{Discussion/Open Problems}

 %----------------------------------------------------------------------------------------
%	REFERENCES
%----------------------------------------------------------------------------------------
\nocite{*} % Print all references regardless of whether they were cited in the poster or not
\bibliographystyle{plain} % Plain referencing style
\bibliography{mybib} % Use the example bibliography file sample.bib


\end{multicols}
\end{document}
