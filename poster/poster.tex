\documentclass[a0,landscape]{a0poster}
\input{preamble.tex} % after the basics
\definecolor{PMS310C}{RGB}{62,192,197} % Official MTU colorway



\begin{document}
%----------------------------------------------------------------------------------------
% HEADER
%----------------------------------------------------------------------------------------
\begin{minipage}[t]{0.35\linewidth}
  \raggedright
  \veryHuge \color{PMS310C} \textbf{Proof-Of-Concept to Production} \color{Black}\\[0.4cm]
  \Huge\textit{Fast Julia Differential Equation Solving}\\[0.8cm]
\end{minipage}
\hfill
\begin{minipage}[t]{0.34\linewidth}
  \raggedright
  {\Large\textbf{Abstract}}\par\smallskip
  {\large
  We compare naïve and optimized Julia implementations for solving
  small to moderate systems of ODEs
  $\vx'(t) = \vf(\vx, t) \, : \, \dim(\vx) \leq 10^2$.
  We demonstrate how a sequence of targeted changes---in-place updates, type stability, and solver selection---can yield order-of-magnitude
  speedups while preserving code clarity and reproducibility. By applying these techniques, we achieve a 10x speedup over the original code and demonstrate how to scale up to larger systems.
  }
\end{minipage}
\hfill
\begin{minipage}[t]{0.18\linewidth}
  \raggedleft
  \red{\textbf{Contact Information:}\\[0.25cm]
  Daniel Henderson\\
  Michigan Technological University\\
  Department of Mathematical Sciences\\
  Houghton, MI\\
  Email: \texttt{dphender@mtu.edu}}
\end{minipage}
\hfill
\begin{minipage}[t]{0.08\linewidth}
  \raggedleft
  \includegraphics[width=\linewidth]{logo.png}
\end{minipage}

\medskip
\hrulefill

%----------------------------------------------------------------------------------------
% BODY (3 COLUMNS)
%----------------------------------------------------------------------------------------
\begin{multicols}{3}

%------------------------ INTRODUCTION ------------------------
\section*{Introduction}

\noindent
Numerical solution of differential equations underpins many scientific and
engineering workflows. Modern simulation pipelines often require millions of
right-hand side evaluations over long horizons or large parameter sweeps, so
even modest inefficiencies can waste substantial time and memory.

\medskip

\noindent
Many users—especially those coming from MATLAB, SciPy, or R—write ODE
right-hand sides in an out-of-place, vectorized style that allocates
temporary arrays, copies data via slicing, and depends on dynamic types or
globals. While convenient for prototypes, this leads to heavy allocation and
garbage collection, and scales poorly for long runs or ensembles. Achieving
``production'' performance therefore requires both an appropriate integrator
and performance-conscious model system code.

\medskip

\noindent
\texttt{DifferentialEquations.jl} (SciML) provides high-performance solvers for
ordinary, stochastic, delay, and partial differential equations and ranks among
the fastest across languages. We distill the SciML optimization tutorial by:
\begin{itemize}
  \item benchmarking naïve vs.\ optimized Julia code on a small nonstiff R\"ossler system;
  \item removing allocations and ensuring type stability via in-place updates,
        \texttt{StaticArrays.jl}, fused broadcasts, BLAS operations, and matrix-free stencils; and
  \item showing how solver choice (explicit Runge--Kutta vs.\ implicit BDF) further
        amplifies these gains, especially for stiff problems.
\end{itemize}
Together, these examples chart a practical path from prototype to production-ready

\medskip

\hrulefill

\medskip

% --- Figure placeholder: overview / pipeline ---
\medskip
\begin{center}
  \includegraphics[width=0.9\linewidth]{pipeline_placeholder.pdf}
  \captionof{figure}{%
    Lorem ipsum dolor sit amet, consectetur adipiscing elit. 
    Placeholder overview diagram of the proof-of-concept to 
    production ODE solver workflow.}
\end{center}

% --- END FIRST COLUMN
\columnbreak

% ------------------------ Ideally: Start of Middle Column ------------------------
\noindent
The Rössler system is a well-known nonstiff autonomous ODE that exhibits chaotic
behavior:
\begin{flalign*}
  \tag{Rössler} \label{eq:rossler}%
  \qquad \vx'(t) = \vf(\vx, t; \vp) = 
  \vect{
    -x_2 - x_3 \\
    x_1 + a x_2 \\
    b + x_1 x_2 - c x_3
  }
    ~ : ~ \begin{cases*}
    \vx = (x_1, x_2, x_3)^\top ~(\text{state variables})\\
    \vp = (a, b, c)^\top ~(\text{parameters}) \\
  \end{cases*}. &&
\end{flalign*}

\medskip


\noindent
Various was to implement \ref{eq:rossler} in Julia:
\begin{lstlisting}[
    language=Julia, 
    caption={``Out-of-place`` implementation allocates new array each call to heap. Here, texttt{@allocated rossler([1.0,1.0,1.0], [0.2,0.2,5.7], 0.0)},
    shows each call allocates 240-bytes}
    label={lst:naive-out-of-place}
  ]
function rossler(vx, vp, t)
    dx1 = -vx[2] - vx[3]
    dx2 = vx[1] + vp[1] * vx[2]
    dx3 = vp[2] + vx[1] * vx[2] - vp[3] * vx[3]
    return [dx1, dx2, dx3]
end
\end{lstlisting}
% In Place
\begin{lstlisting}[
    language=Julia, 
    caption={``In-place`` ODE system, utilizing cache array (``pass-by-reference``)
    ``vx`` to store the derivatives.},
    label={lst:in-place}
  ]
function rossler!(vx, vp, t)
    vx[1] = -vx[2] - vx[3]
    vx[2] = vx[1] + vp[1] * vx[2]
    vx[3] = vp[2] + vx[1] * vx[2] - vp[3] * vx[3]
end
\end{lstlisting}\todo{show lowered code}

\begin{lstlisting}[
    language=Julia, 
    caption={Optimal `in-place`` using static arrays allocated at compile time.},
    label={lst:static}
  ]
  function rossler!(vx, vp, t)
    vx[1] = -vx[2] - vx[3]
    vx[2] = vx[1] + vp[1] * vx[2]
    vx[3] = vp[2] + vx[1] * vx[2] - vp[3] * vx[3]
  end
\end{lstlisting}\todo{show lowered code}

\noindent
Maecenas faucibus mollis interdum. Vivamus sagittis lacus vel augue laoreet
rutrum faucibus dolor auctor. Integer posuere erat a ante venenatis dapibus
posuere velit aliquet.

%------------------------ RESULTS / ANALYSIS ------------------------
\section*{Results/Analysis}

\noindent
Lorem ipsum dolor sit amet, consectetur adipiscing elit. Mauris fermentum
facilisis ligula, id hendrerit lectus gravida sed. Vestibulum ante ipsum primis
in faucibus orci luctus et ultrices posuere cubilia curae; Sed aliquet, erat
vel volutpat tempor, urna lorem vehicula elit, ut facilisis neque enim sed
massa. Suspendisse potenti.

\medskip

% --- Figure placeholder: speedup plot ---
\begin{center}
  \includegraphics[width=0.95\linewidth]{speedup_placeholder.pdf}
  \captionof{figure}{%
    Lorem ipsum dolor sit amet, consectetur adipiscing elit. 
    Placeholder benchmark plot comparing naïve and optimized 
    Julia implementations for a Rössler-type ODE system.}
\end{center}

\medskip

\noindent
Curabitur sit amet convallis urna. Aenean ultricies, mauris at dignissim
egestas, metus orci rhoncus urna, vitae tempor elit arcu ut orci. Integer
ullamcorper, ligula in lacinia pretium, lacus lorem sollicitudin nisi, non
elementum velit lectus at ipsum.

\medskip

% --- Table placeholder: benchmark numbers ---
\begin{center}
  \begin{tabular}{lccc}
    \toprule
    \textbf{Method} & \textbf{Runtime (rel.)} & \textbf{Allocations (rel.)} & \textbf{Speedup} \\
    \midrule
    Naïve Julia      & 1.00 & 1.00 & $1.0\times$ \\
    Optimized Julia  & 0.12 & 0.05 & $8.3\times$ \\
    SciML baseline   & 0.10 & 0.04 & $10.0\times$ \\
    \bottomrule
  \end{tabular}
  \captionof{table}{%
    Lorem ipsum dolor sit amet, consectetur adipiscing elit. 
    Placeholder benchmark table comparing relative runtime, 
    memory allocations, and overall speedup factors.}
\end{center}

\medskip

\noindent
Pellentesque habitant morbi tristique senectus et netus et malesuada fames ac
turpis egestas. Sed posuere consectetur est at lobortis. Vestibulum id ligula
porta felis euismod semper.

%------------------------ DISCUSSION / OPEN PROBLEMS ------------------------
\section*{Discussion / Open Problems}

\noindent
Lorem ipsum dolor sit amet, consectetur adipiscing elit. Integer in lacus nec
odio commodo condimentum. Pellentesque vitae elit at lacus vulputate posuere.
Morbi laoreet, justo at varius hendrerit, justo nisi consequat nisl, non
tincidunt neque augue quis leo. Sed viverra urna id dui pulvinar, non gravida
nisl viverra.

\medskip

% --- Figure placeholder: future work / design sketch ---
\begin{center}
  \includegraphics[width=0.85\linewidth]{futurework_placeholder.pdf}
  \captionof{figure}{%
    Lorem ipsum dolor sit amet, consectetur adipiscing elit. 
    Placeholder conceptual diagram for future extensions 
    (e.g., GPU acceleration, parameter sweeps, or PDE MOL extensions).}
\end{center}

\medskip

\noindent
Suspendisse ut leo nec nibh ultricies finibus. Sed euismod, nisl vel tempor
mattis, metus arcu pulvinar turpis, sed luctus odio tellus id massa. Donec in
lectus vitae sapien iaculis volutpat quis in neque. Cras mattis consectetur
purus sit amet fermentum.

%------------------------ REFERENCES ------------------------
\nocite{*}
\bibliographystyle{plain}
\bibliography{mybib}

\end{multicols}
\end{document}
