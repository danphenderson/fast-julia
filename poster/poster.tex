\documentclass[a0,landscape]{a0poster}
\usepackage{times}
\input{preamble.tex} % after the basics

\definecolor{PMS310C}{RGB}{62,192,197} % Official MTU colorway
\begin{document}

%----------------------------------------------------------------------------------------
% HEADER
%----------------------------------------------------------------------------------------
\begin{minipage}[b]{0.55\linewidth}
  \raggedright
  \veryHuge \color{PMS310C} \textbf{Proof-Of-Concept to Production} \color{Black}\\[0.4cm]
  \Huge\textit{Fast Julia Differential Equation Solving}\\[0.8cm]
\end{minipage}
\hfill
\begin{minipage}[b]{0.23\linewidth}
  \raggedright
  \red{\Large \textbf{Contact Information:}\\[0.25cm]
  Daniel Henderson\\
  Michigan Technological University\\
  Department of Mathematical Sciences\\
  Houghton, MI\\
  Email: \texttt{dphender@mtu.edu}}
\end{minipage}
\hfill
\begin{minipage}[b]{0.18\linewidth}
  \raggedleft
  \vspace{0.4cm}
  \includegraphics[width=0.9\linewidth]{logo.png}\\[0.6cm]
\end{minipage}

\vspace{1cm}
\hrulefill

%----------------------------------------------------------------------------------------
% BODY (3 COLUMNS)
%----------------------------------------------------------------------------------------

\begin{multicols}{3}

%------------------------ ABSTRACT ------------------------
\begin{abstract}
  \noindent
  We compare naïve and optimized Julia implementations for solving 
  small to moderate systems of ordinary differential equations 
  (up to $\mathcal{O}(10^2)$ state variables). Using 
  \texttt{DifferentialEquations.jl} and standard performance tools, 
  we demonstrate how a sequence of targeted changes---in-place updates, 
  type stability, and solver selection---can yield order-of-magnitude 
  speedups while preserving code clarity and reproducibility.
\end{abstract}

%------------------------ INTRODUCTION ------------------------
\section*{Introduction}

\noindent
Numerical solutions of differential equations underpins a wide range
of scientific and engineering workflows. Modern simulation pipelines 
routinely require millions of right-hand side evaluations, long time 
horizons, or large parameter sweeps. In this setting, even modest 
inefficiencies in user code can translate into substantial wasted 
compute time and memory.

\medskip

\noindent
Many users, especially those coming from MATLAB, SciPy, or R, naturally
express ODE systems in an out-of-place, vectorized style that allocates
temporary arrays at each function call, performs slicing that copies
data, and relies on dynamic types and global variables. This approach
is convenient for proof-of-concept scripts, but suffers from excessive 
memory allocation and garbage collection, and consequently poor scaling
for long simulations or ensembles. Acheiving ``production'' performance 
requires not only choosing an appropriate integrator, but also writing the model 
equations in a performance-conscious style.

\medskip

\noindent
The Julia package \texttt{DifferentialEquations.jl}, part of the SciML
ecosystem, provides state-of-the-art algorithms for ordinary, stochastic,
delay, and partial differential equations, and benchmarks among the
fastest solver libraries across languages. Herein, we distill and extend
the SciML tutorial on optimizing differential equation code by:
\begin{itemize}
  \item contrasting naïve and optimized Julia implementations on a small
        nonstiff Rossler system.
  \item applying a sequence of user-side optimizations---in-place
        updates, \texttt{StaticArrays.jl} for small state vectors,
        broadcast fusion, BLAS-based matrix operations, and
        matrix-free stencils---to make the right-hand side essentially
        non-allocating and type-stable; and
  \item demonstrating how appropriate solver choice (e.g., explicit
        Runge--Kutta vs.\ implicit BDF methods) further amplifies these
        gains, especially for stiff problems.
\end{itemize}
Together, these examples illustrate a practical path from
proof-of-concept to production-ready Julia differential equation solvers,
with order-of-magnitude improvements in runtime and memory usage while
maintaining clear, reproducible code.

% --- Figure placeholder: overview / pipeline ---
\medskip
\begin{center}
  \includegraphics[width=0.9\linewidth]{pipeline_placeholder.pdf}
  \captionof{figure}{%
    Lorem ipsum dolor sit amet, consectetur adipiscing elit. 
    Placeholder overview diagram of the proof-of-concept to 
    production ODE solver workflow.}
\end{center}
\medskip

\noindent
Maecenas faucibus mollis interdum. Vivamus sagittis lacus vel augue laoreet
rutrum faucibus dolor auctor. Integer posuere erat a ante venenatis dapibus
posuere velit aliquet.

%------------------------ RESULTS / ANALYSIS ------------------------
\section*{Results/Analysis}

\noindent
Lorem ipsum dolor sit amet, consectetur adipiscing elit. Mauris fermentum
facilisis ligula, id hendrerit lectus gravida sed. Vestibulum ante ipsum primis
in faucibus orci luctus et ultrices posuere cubilia curae; Sed aliquet, erat
vel volutpat tempor, urna lorem vehicula elit, ut facilisis neque enim sed
massa. Suspendisse potenti.

\medskip

% --- Figure placeholder: speedup plot ---
\begin{center}
  \includegraphics[width=0.95\linewidth]{speedup_placeholder.pdf}
  \captionof{figure}{%
    Lorem ipsum dolor sit amet, consectetur adipiscing elit. 
    Placeholder benchmark plot comparing naïve and optimized 
    Julia implementations for a Rossler-type ODE system.}
\end{center}

\medskip

\noindent
Curabitur sit amet convallis urna. Aenean ultricies, mauris at dignissim
egestas, metus orci rhoncus urna, vitae tempor elit arcu ut orci. Integer
ullamcorper, ligula in lacinia pretium, lacus lorem sollicitudin nisi, non
elementum velit lectus at ipsum.

\medskip

% --- Table placeholder: benchmark numbers ---
\begin{center}
  \begin{tabular}{lccc}
    \toprule
    \textbf{Method} & \textbf{Runtime (rel.)} & \textbf{Allocations (rel.)} & \textbf{Speedup} \\
    \midrule
    Naïve Julia      & 1.00 & 1.00 & $1.0\times$ \\
    Optimized Julia  & 0.12 & 0.05 & $8.3\times$ \\
    SciML baseline   & 0.10 & 0.04 & $10.0\times$ \\
    \bottomrule
  \end{tabular}
  \captionof{table}{%
    Lorem ipsum dolor sit amet, consectetur adipiscing elit. 
    Placeholder benchmark table comparing relative runtime, 
    memory allocations, and overall speedup factors.}
\end{center}

\medskip

\noindent
Pellentesque habitant morbi tristique senectus et netus et malesuada fames ac
turpis egestas. Sed posuere consectetur est at lobortis. Vestibulum id ligula
porta felis euismod semper.

%------------------------ DISCUSSION / OPEN PROBLEMS ------------------------
\section*{Discussion / Open Problems}

\noindent
Lorem ipsum dolor sit amet, consectetur adipiscing elit. Integer in lacus nec
odio commodo condimentum. Pellentesque vitae elit at lacus vulputate posuere.
Morbi laoreet, justo at varius hendrerit, justo nisi consequat nisl, non
tincidunt neque augue quis leo. Sed viverra urna id dui pulvinar, non gravida
nisl viverra.

\medskip

% --- Figure placeholder: future work / design sketch ---
\begin{center}
  \includegraphics[width=0.85\linewidth]{futurework_placeholder.pdf}
  \captionof{figure}{%
    Lorem ipsum dolor sit amet, consectetur adipiscing elit. 
    Placeholder conceptual diagram for future extensions 
    (e.g., GPU acceleration, parameter sweeps, or PDE MOL extensions).}
\end{center}

\medskip

\noindent
Suspendisse ut leo nec nibh ultricies finibus. Sed euismod, nisl vel tempor
mattis, metus arcu pulvinar turpis, sed luctus odio tellus id massa. Donec in
lectus vitae sapien iaculis volutpat quis in neque. Cras mattis consectetur
purus sit amet fermentum.

%------------------------ REFERENCES ------------------------
\nocite{*}
\bibliographystyle{plain}
\bibliography{mybib}

\end{multicols}
\end{document}
