\documentclass[a0,landscape]{a0poster}
\input{preamble.tex} % after the basics
\definecolor{PMS310C}{RGB}{62,192,197} % Official MTU colorway



\begin{document}
\begin{tikzpicture}[remember picture,overlay]
  \node[anchor=north east, xshift=-5mm, yshift=-5mm] 
    at (current page.north east) {\includegraphics[height=7cm]{logo.png}};
\end{tikzpicture}

%----------------------------------------------------------------------------------------
% HEADER
%----------------------------------------------------------------------------------------
\vspace*{-2cm} % reduce magnitude if anything touches the top edge

{\setlength{\tabcolsep}{0pt}%
\renewcommand{\arraystretch}{1.05}%

\noindent
\begin{minipage}[t]{0.34\textwidth}
  {\Huge\bfseries\color{PMS310C} POC to Production}\\[1ex]
  {\Large\itshape Fast Julia Differential Equation Solving}\\[2ex]
  {\large Daniel Henderson}\\
  {\normalsize Michigan Technological University}\\
  {\small\ttfamily dphender@mtu.edu}
\end{minipage}%
\begin{minipage}[t]{0.24\textwidth}\vspace{0pt}
{\Large\textbf{Abstract}}
{\normalsize
We compare naïve and optimized Julia implementations for solving
small to moderate systems of ODEs
$\vx'(t) = \vf(\vx, t) \, : \, \dim(\vx) \leq 10^2$.
We demonstrate how a sequence of targeted changes---in-place updates, type stability, and
solver selection---can yield order-of-magnitude speedups while preserving code clarity and
reproducibility. By applying these techniques, we achieve a 10x speedup over the original
code and demonstrate how to scale up to larger systems.}
\end{minipage}
}

\vspace{0.20cm}
\hrulefill
\vspace{0.20cm}

%----------------------------------------------------------------------------------------
% BODY (3 COLUMNS)
%----------------------------------------------------------------------------------------
% ...existing code...

%----------------------------------------------------------------------------------------
% BODY (3 COLUMNS)
%----------------------------------------------------------------------------------------
\begin{paracol}{3}
  \setcolumnwidth{0.28\textwidth,0.44\textwidth,0.28\textwidth}


%------------------------ INTRODUCTION ------------------------
\section*{Introduction}

\noindent
Numerical solution of differential equations underpins many scientific and
engineering workflows. Modern simulation pipelines often require millions of
right-hand side evaluations over long horizons or large parameter sweeps, so
even modest inefficiencies can waste substantial time and memory.

\medskip

\noindent
We demonstrate how to optimize a simple ODE solver pipeline from prototype to
production using Julia, a high-level, high-performance language for numerical
computation \cite{Julia-2017}. Our first case study focuses on the chaotic
system
\begin{flalign*}
  \tag{Rössler} \label{eq:rossler}
  \qquad \vx'(t) = \vf(\vx, t; \vp) = 
  \vect{
    -x_2 - x_3 \\
    x_1 + a x_2 \\
    b + x_3 (x_1 - c) \\
  }
    ~ : ~ \begin{cases*}
    \vx = (x_1, x_2, x_3)^\top ~(\text{state variables})\\
    \vp = (a, b, c)^\top ~(\text{parameters}) \\
  \end{cases*}, &&
\end{flalign*}
introduced as a simplification of the Lorenz attractor
by Otto Rössler in 1976 \cite{Rössler-1976}.

\medskip

\noindent
We seek numerical solution $\vx(t) ~ \forall t \in [t_0, t_f]$ subject to $\vx(t_0) = \vx_0$,
with parameters $\vp$ fixed via \texttt{DifferentialEquations.jl} (by SciML organization)--
performant ODE solving \cite{DifferentialEquations.jl-2017},
built on the Julia language design \cite{Julia-2017}. We build on the SciML 
tutorial~\cite{SciMLTutorial} for the Rössler system, but with a focus on code
optimization and performance tuning.

\medskip

\noindent
\textbf{Case Study 1: Rossler System.}
\begin{itemize}
  \item naive ``out-of-place'' Julia code;
  \item optimization with in-place updates and stack-allocated arrays;
\end{itemize}
\textbf{Case Study 2: Parameter Sweep of Rossler System}
\begin{itemize}
  \item parameter sweep over $\vp$ using fused broadcasts, BLAS operations, and
        matrix-free stencils; and
  \item demonstrating how solver choice (explicit Runge--Kutta vs.\ implicit BDFk) further
        amplifies these gains, especially for stiff problems.
\end{itemize}

\medskip

\noindent
``Production'' performance is acheived by optimization of the umderlying model
code -- our focus herein, rather than on the integrator itself.

\medskip


\medskip
\hrulefill
\medskip

% --- Figure placeholder: overview / pipeline ---
\medskip
\begin{center}
  \includegraphics[width=0.9\linewidth]{pipeline_placeholder.pdf}
  \captionof{figure}{%
    Lorem ipsum dolor sit amet, consectetur adipiscing elit. 
    Placeholder overview diagram of the proof-of-concept to 
    production ODE solver workflow.}
\end{center}

% --- END FIRST COLUMN
\switchcolumn


% ------------------------ Ideally: Start of Middle Column ------------------------
\noindent
\begin{minipage}{\linewidth}
\centering
\begin{minipage}[t]{0.475\linewidth} % Left: out-of-place
\begin{lstlisting}[
  language=Julia,
  linewidth=\linewidth,
  breaklines=true,
  caption={``Out-of-place`` allocates a new heap array each call.},
  label={lst:naive-out-of-place}
]
function rossler(vx, vp, t)
    dx1 = -vx[2] - vx[3]
    dx2 =  vx[1] + vp[1] * vx[2]
    dx3 =  vp[2] + vx[3] * (vx[1] - vp[3])
    return [dx1, dx2, dx3]
end
\end{lstlisting}
\td{\small \texttt{@allocated rossler([1.0,1.0,1.0], (0.1,0.1,12), 0.0)}
  reports 240 bytes per call (typical).}
\end{minipage}
\hfill
\begin{minipage}[t]{0.475\linewidth} % Right: in-place
\begin{lstlisting}[
  language=Julia,
  linewidth=\linewidth,
  breaklines=true,
  caption={``In-place`` ODE system, utilizing cache array (`vx` passed-by-reference)
  for state values. Avoids per-call allocations and typically fastest for `Vector`-based states.},
  label={lst:in-place}
]
function rossler!(dx, vx, vp, t)
    x1 = vx[1]
    dx[1] = -vx[2] - vx[3]
    dx[2] =  x1 + vp[1] * vx[2]
    dx[3] =  vp[2] + vx[3] * (x1 - vp[3])
    return nothing
end
\end{lstlisting}
\end{minipage}
\vspace{0.75em}
\begin{lstlisting}[
  language=Julia,
  linewidth=\linewidth,
  breaklines=true,
  caption={Fixed-size, stack-allocated \texttt{\cite{StaticArrays.jl-2017}} version.},
  label={lst:static}
]
function rossler_static(vx, vp, t)
    x1 = vx[1]
    dx1 = -vx[2] - vx[3]
    dx2 =  x1 + vp[1] * vx[2]
    dx3 =  vp[2] + vx[3] * (x1 - vp[3])
    return @SVector [dx1, dx2, dx3]
end
\end{lstlisting}
\end{minipage}

\noindent
Maecenas faucibus mollis interdum. Vivamus sagittis lacus vel augue laoreet
rutrum faucibus dolor auctor. Integer posuere erat a ante venenatis dapibus
posuere velit aliquet.


%------------------------ RESULTS / ANALYSIS ------------------------
\section*{Results/Analysis}

\noindent
Lorem ipsum dolor sit amet, consectetur adipiscing elit. Mauris fermentum
facilisis ligula, id hendrerit lectus gravida sed. Vestibulum ante ipsum primis
in faucibus orci luctus et ultrices posuere cubilia curae; Sed aliquet, erat
vel volutpat tempor, urna lorem vehicula elit, ut facilisis neque enim sed
massa. Suspendisse potenti.

\medskip

% --- Figure placeholder: speedup plot ---'
\noindent
\begin{minipage}{\linewidth}
\centering
\begin{minipage}[t]{0.475\linewidth} % Left: out-of-place
  \includegraphics[width=0.95\linewidth]{rk4_time_normalized.png}
  \captionof{figure}{%
    Lorem ipsum dolor sit amet, consectetur adipiscing elit. 
    Placeholder benchmark plot comparing naïve and optimized 
    Julia implementations for a Rössler-type ODE system.}
\end{minipage}
\hfill
\begin{minipage}[t]{0.475\linewidth} % Right: in-place
  \includegraphics[width=0.95\linewidth]{rk4_time_normalized.png}
  \captionof{figure}{%
    Lorem ipsum dolor sit amet, consectetur adipiscing elit. 
    Placeholder benchmark plot comparing naïve and optimized 
    Julia implementations for a Rössler-type ODE system.}
\end{minipage}
\end{minipage}

% --- END MIDDLE COLUMN
\switchcolumn

\medskip

\noindent
Curabitur sit amet convallis urna. Aenean ultricies, mauris at dignissim
egestas, metus orci rhoncus urna, vitae tempor elit arcu ut orci. Integer
ullamcorper, ligula in lacinia pretium, lacus lorem sollicitudin nisi, non
elementum velit lectus at ipsum.

\medskip

% --- Table placeholder: benchmark numbers ---
\begin{center}
  \begin{tabular}{lccc}
    \toprule
    \textbf{Method} & \textbf{Runtime (rel.)} & \textbf{Allocations (rel.)} & \textbf{Speedup} \\
    \midrule
    Naïve Julia      & 1.00 & 1.00 & $1.0\times$ \\
    Optimized Julia  & 0.12 & 0.05 & $8.3\times$ \\
    SciML baseline   & 0.10 & 0.04 & $10.0\times$ \\
    \bottomrule
  \end{tabular}
  \captionof{table}{%
    Lorem ipsum dolor sit amet, consectetur adipiscing elit. 
    Placeholder benchmark table comparing relative runtime, 
    memory allocations, and overall speedup factors.}
\end{center}

\medskip

\noindent
Pellentesque habitant morbi tristique senectus et netus et malesuada fames ac
turpis egestas. Sed posuere consectetur est at lobortis. Vestibulum id ligula
porta felis euismod semper.

%------------------------ DISCUSSION / OPEN PROBLEMS ------------------------
\section*{Discussion / Open Problems}

\noindent
Lorem ipsum dolor sit amet, consectetur adipiscing elit. Integer in lacus nec
odio commodo condimentum. Pellentesque vitae elit at lacus vulputate posuere.
Morbi laoreet, justo at varius hendrerit, justo nisi consequat nisl, non
tincidunt neque augue quis leo. Sed viverra urna id dui pulvinar, non gravida
nisl viverra.

\medskip

% --- Figure placeholder: future work / design sketch ---
\begin{center}
  \includegraphics[width=0.85\linewidth]{futurework_placeholder.pdf}
  \captionof{figure}{%
    Lorem ipsum dolor sit amet, consectetur adipiscing elit. 
    Placeholder conceptual diagram for future extensions 
    (e.g., GPU acceleration, parameter sweeps, or PDE MOL extensions).}
\end{center}

\medskip

\noindent
Suspendisse ut leo nec nibh ultricies finibus. Sed euismod, nisl vel tempor
mattis, metus arcu pulvinar turpis, sed luctus odio tellus id massa. Donec in
lectus vitae sapien iaculis volutpat quis in neque. Cras mattis consectetur
purus sit amet fermentum.

%------------------------ REFERENCES ------------------------
\nocite{*}
\bibliographystyle{plain}
\bibliography{mybib}

\end{paracol}
\end{document}
