\documentclass[a0,landscape]{a0poster}
\input{preamble.tex} % after the basics
\definecolor{PMS310C}{RGB}{62,192,197} % Official MTU colorway

\begin{document}
\begin{tikzpicture}[remember picture,overlay]
  \node[anchor=north east, xshift=-5mm, yshift=-5mm] 
    at (current page.north east) {\includegraphics[height=7cm]{logo.png}};
\end{tikzpicture}

%----------------------------------------------------------------------------------------
% HEADER
%----------------------------------------------------------------------------------------
\vspace*{-2cm} % reduce magnitude if anything touches the top edge

{\setlength{\tabcolsep}{0pt}%
\renewcommand{\arraystretch}{1.05}%

\noindent
\begin{minipage}[t]{0.34\textwidth}
  {\Huge\bfseries\color{PMS310C} POC to Production}\\[1ex]
  {\Large\itshape Fast Julia Differential Equation Solving}\\[2ex]
  {\large Daniel Henderson}\\
  {\normalsize Michigan Technological University}\\
  {\small\ttfamily dphender@mtu.edu}
\end{minipage}%
\begin{minipage}[t]{0.24\textwidth}\vspace{0pt}
{\Large\textbf{Abstract}}
{\normalsize
We compare naïve and optimized Julia implementations for solving
small to moderate systems of ordinary differential equations (ODEs),
$\vx'(t) = \vf(\vx, t)$ with $\dim(\vx) \leq 10^2$.
We demonstrate how a sequence of targeted changes—in-place updates, type stability and
solver selection—can yield order-of-magnitude speedups while preserving code clarity and
reproducibility.  By applying these techniques we achieve a tenfold speedup over the original
code and demonstrate how to scale up to larger systems.}
\end{minipage}
}

\vspace{0.20cm}
\hrulefill
\vspace{0.20cm}

%----------------------------------------------------------------------------------------
% BODY (3 COLUMNS)
%----------------------------------------------------------------------------------------
\begin{paracol}{3}
  \setcolumnwidth{0.28\textwidth,0.44\textwidth,0.28\textwidth}

%------------------------ INTRODUCTION ------------------------
\section*{Introduction}

\noindent
ODE workflows often involve ensembles: many solves across parameters, initial conditions,
or time horizons. In that regime, per-call overhead in 
$f(x,p,t)$—especially allocations—dominates and can make otherwise
simple experiments impractically slow.”

\medskip

\noindent
Targeted changes to model code-—in-place updates, type stability, and
stack-vs.-heap allocations, etc. combined with the right choice of solver can
yield order-of-magnitude speedups while keeping the code readable and
extensible. In our case, we achieve a \red{XX} speedup from a naïve to an 
optimized Julia implementation of the Rössler ODE system:
\begin{flalign*}
  \tag{Rössler} \label{eq:rossler}
  \quad \vx'(t) = \vf(\vx,\, \vp, \, t) = 
  \vect{
    -x_2 \,-\, x_3 \\
    x_1 \,+\, a x_2 \\
    b \,+\, x_3 (x_1 \,-\, c) \\
  }
    ~ : ~ \begin{cases*}
    \vx = (x_1, \, x_2, \, x_3)^\top \text{ (state variables)}\\
    \vp = (a, \, b, \, c)^\top \text{ (parameters)} 
  \end{cases*}, &&
\end{flalign*}

\medskip

\noindent
We seek $\vx(t)$ for all $t \in [t_0, t_f]$ subject to the initial condition $\vx(t_0)=\vx_0$, with fixed
parameters $\vp$. ``Production'' performance is achieved using \texttt{DifferentialEquations.jl} from the SciML
ecosystem \cite{DifferentialEquations.jl-2017}, building on the SciML techniques
\cite{SciMLTutorial} for optimizing the model code $\vf$.

\medskip

\noindent
\textbf{Case Study.}  We compare out-of-place and in-place Julia implementations
of the Rössler system \eqref{eq:rossler}, together with a type-stable fixed-size variant:
\begin{itemize}
  \item naïve out-of-place code allocates a new array at each right-hand-side call, triggering garbage collection;
  \item optimized in-place code eliminates per-call allocations for \texttt{Vector}-based states; and
  \item a \texttt{StaticArrays.jl}/\texttt{SVector} implementation enables stack allocation, often fastest for very small systems.
\end{itemize}

\noindent
\textbf{Future of SciML is Symbolic--Numeric.}  Symbolic manipulation simplifies \eqref{eq:rossler} via substitution and highlights structure (two linear equations in $\vx$ and one quadratic).

\medskip
\hrulefill
\medskip

% --- Figure: Pipeline Overview (TikZ) ---
\medskip
\begin{center}

% Title formatter (this is what \stagetitle was *supposed* to be)
\providecommand{\PipelineStageTitle}[1]{%
  {\bfseries\Large\color{PMS310C}#1}%
}

\begin{tikzpicture}[
  font=\normalsize,
  stage/.style={
    draw=black, rounded corners=2.5mm, thick, fill=black!2,
    inner sep=7pt, text width=0.86\linewidth
  },
  pill/.style={
    draw=black!50, rounded corners=1.2mm, fill=white,
    font=\ttfamily\scriptsize, inner xsep=4pt, inner ysep=2pt
  },
  iconbox/.style={
    draw=black!55, rounded corners=1mm, thick, fill=white,
    minimum width=1.25cm, minimum height=1.25cm
  },
  arrow/.style={-Latex, thick},
  feedback/.style={-Latex, thick, dashed, draw=black!70},
  node distance=7mm
]
% Reserve horizontal padding inside each stage for the icon.
\def\IconPad{1.95cm}

% ----------------------------
% Simple inline “icons” (pure TikZ)
% ----------------------------
\providecommand{\PipelineCodeIcon}{%
  \tikz[scale=0.34, line width=0.9pt]{
    \draw[rounded corners=0.7pt] (0,0) rectangle (1.2,1.2);
    \draw (0.18,0.95)--(1.02,0.95);
    \draw (0.18,0.72)--(0.82,0.72);
    \draw (0.18,0.50)--(1.02,0.50);
    \draw (0.18,0.27)--(0.70,0.27);
    \draw (0.18,0.12)--(0.55,0.12);
  }%
}
\providecommand{\PipelineGearIcon}{%
  \tikz[scale=0.34, line width=0.9pt]{
    \draw (0.6,0.6) circle (0.34);
    \foreach \a in {0,30,...,330}{
      \draw[shift={(0.6,0.6)}, rotate=\a] (0.34,0) -- (0.50,0);
    }
    \draw (0.6,0.6) circle (0.12);
  }%
}
\providecommand{\PipelineStopwatchIcon}{%
  \tikz[scale=0.34, line width=0.9pt]{
    \draw (0.6,0.55) circle (0.38);
    \draw (0.45,0.98) rectangle (0.75,1.12);
    \draw (0.56,0.55) -- (0.56,0.80);
    \draw (0.56,0.55) -- (0.80,0.55);
    \draw (0.78,0.92) -- (0.98,1.05);
  }%
}
\providecommand{\PipelineBarsIcon}{%
  \tikz[scale=0.34, line width=0.9pt]{
    \draw (0.15,0.15) -- (0.15,1.10);
    \draw (0.15,0.15) -- (1.10,0.15);
    \filldraw (0.30,0.15) rectangle (0.48,0.55);
    \filldraw (0.55,0.15) rectangle (0.73,0.85);
    \filldraw (0.80,0.15) rectangle (0.98,0.42);
  }%
}
\providecommand{\PipelineChecklistIcon}{%
  \tikz[scale=0.34, line width=0.9pt]{
    \draw[rounded corners=0.7pt] (0,0) rectangle (1.2,1.2);
    \draw (0.18,0.86) -- (0.28,0.76) -- (0.45,0.94);
    \draw (0.55,0.86) -- (1.05,0.86);
    \draw (0.18,0.58) -- (0.28,0.48) -- (0.45,0.66);
    \draw (0.55,0.58) -- (1.05,0.58);
    \draw (0.18,0.30) -- (0.28,0.20) -- (0.45,0.38);
    \draw (0.55,0.30) -- (1.05,0.30);
  }%
}

% ----------------------------
% Stages (vertical pipeline)
% ----------------------------
\node[stage] (s1) {%
  \hspace*{\IconPad}\PipelineStageTitle{Stage 1: Mathematical Model \& Code}\par
  \hspace*{\IconPad}\normalsize Define the R\"ossler RHS in Julia.\par
  \hspace*{\IconPad}\scriptsize
    \tikz[baseline=-0.6ex]{\node[pill]{na\"ive};}\,
    \tikz[baseline=-0.6ex]{\node[pill]{in-place};}\,
    \tikz[baseline=-0.6ex]{\node[pill]{StaticArrays/SVector};}%
};

\node[stage, below=of s1] (s2) {%
  \hspace*{\IconPad}\PipelineStageTitle{Stage 2: Compilation \& Execution}\par
  \hspace*{\IconPad}\normalsize Julia JIT compiles hot paths to native code.\par
  \hspace*{\IconPad}\scriptsize (Warm-up compile excluded from measurements.)%
};

\node[stage, below=of s2] (s3) {%
  \hspace*{\IconPad}\PipelineStageTitle{Stage 3: Benchmarking \& Profiling}\par
  \hspace*{\IconPad}\normalsize Measure runtime \& allocations per implementation.\par
  \hspace*{\IconPad}\scriptsize \texttt{@benchmark/@btime} + \texttt{@allocated} + type checks%
};

\node[stage, below=of s3] (s4) {%
  \hspace*{\IconPad}\PipelineStageTitle{Stage 4: Visualization of Results}\par
  \hspace*{\IconPad}\normalsize Summarize performance and solution behavior.\par
  \hspace*{\IconPad}\scriptsize
    \tikz[baseline=-0.6ex]{\node[pill]{time};}\,
    \tikz[baseline=-0.6ex]{\node[pill]{allocs};}\,
    \tikz[baseline=-0.6ex]{\node[pill]{trajectory/GIF};}%
};

\node[stage, below=of s4] (s5) {%
  \hspace*{\IconPad}\PipelineStageTitle{Stage 5: Insights \& Optimization Cycle}\par
  \hspace*{\IconPad}\normalsize Convert findings into actionable improvements.\par
  \hspace*{\IconPad}\scriptsize e.g., eliminate allocations; choose data layout; select solver%
};

% Icons
\node[iconbox, anchor=west] at ([xshift=6pt]s1.west) {\PipelineCodeIcon};
\node[iconbox, anchor=west] at ([xshift=6pt]s2.west) {\PipelineGearIcon};
\node[iconbox, anchor=west] at ([xshift=6pt]s3.west) {\PipelineStopwatchIcon};
\node[iconbox, anchor=west] at ([xshift=6pt]s4.west) {\PipelineBarsIcon};
\node[iconbox, anchor=west] at ([xshift=6pt]s5.west) {\PipelineChecklistIcon};

% Forward arrows
\draw[arrow] (s1) -- (s2);
\draw[arrow] (s2) -- (s3);
\draw[arrow] (s3) -- (s4);
\draw[arrow] (s4) -- (s5);

% Feedback loop
\draw[feedback] (s5.east) -- ++(10mm,0) |- (s1.east);

\end{tikzpicture}

{\captionsetup{hypcap=false}%
\captionof{figure}{Pipeline overview: model/code $\rightarrow$ JIT compile $\rightarrow$ benchmark/profile $\rightarrow$ visualize $\rightarrow$ iterate.}}

\end{center}

% --- END FIRST COLUMN
\switchcolumn

% ------------------------ Middle Column ------------------------
\section*{Case Study: Optimizing the Rössler ODE System}

\noindent
\begin{minipage}{\linewidth}
\centering
\begin{minipage}[t]{0.475\linewidth} % Left: out-of-place
\begin{lstlisting}[
  language=Julia,
  linewidth=\linewidth,
  breaklines=true,
  caption={``Out-of-place'' allocates a new heap array at each call.},
  label={lst:naive-out-of-place}
]
function rossler(vx, vp, t)
    dx1 = -vx[2] - vx[3]
    dx2 =  vx[1] + vp[1] * vx[2]
    dx3 =  vp[2] + vx[3] * (vx[1] - vp[3])
    return [dx1, dx2, dx3]
end
\end{lstlisting}
\td{\small \texttt{@allocated rossler([1.0,1.0,1.0], (0.1,0.1,12), 0.0)}
  reports 240 bytes per call (typical).}
\end{minipage}
\hfill
\begin{minipage}[t]{0.475\linewidth} % Right: in-place
\begin{lstlisting}[
  language=Julia,
  linewidth=\linewidth,
  breaklines=true,
  caption={``In-place'' ODE system utilizing a cache array (`vx` passed by reference) for state values.  Avoids per-call allocations and is typically fastest for `Vector`-based states.},
  label={lst:in-place}
]
function rossler!(dx, vx, vp, t)
    x1 = vx[1]
    dx[1] = -vx[2] - vx[3]
    dx[2] =  x1 + vp[1] * vx[2]
    dx[3] =  vp[2] + vx[3] * (x1 - vp[3])
    return nothing
end
\end{lstlisting}
\end{minipage}
\vspace{0.75em}
\begin{lstlisting}[
  language=Julia,
  linewidth=\linewidth,
  breaklines=true,
  caption={Fixed-size, stack-allocated version using \texttt{StaticArrays}.},
  label={lst:static}
]
function rossler_static(vx, vp, t)
    x1 = vx[1]
    dx1 = -vx[2] - vx[3]
    dx2 =  x1 + vp[1] * vx[2]
    dx3 =  vp[2] + vx[3] * (x1 - vp[3])
    return @SVector [dx1, dx2, dx3]
end
\end{lstlisting}
\end{minipage}

\noindent
\subsection*{Results and Analysis}

% --- Figures: speedup plot ---
\noindent
\begin{minipage}{\linewidth}
\centering
\begin{minipage}[t]{0.475\linewidth} % Left figure
  \includegraphics[width=0.95\linewidth]{rk4_time_normalized_log10.png}
  {\captionsetup{hypcap=false}%
  \captionof{figure}{Placeholder benchmark plot comparing naïve and optimized Julia implementations for a Rössler-type ODE system.}}
\end{minipage}
\hfill
\begin{minipage}[t]{0.475\linewidth} % Right figure
  \includegraphics[width=0.95\linewidth]{rk4_time_normalized_log10.png}
  {\captionsetup{hypcap=false}%
  \captionof{figure}{Placeholder benchmark plot comparing naïve and optimized Julia implementations for a Rössler-type ODE system.}}
\end{minipage}
\end{minipage}
% --- Table placeholder: benchmark numbers ---
\begin{center}
  \begin{tabular}{lccc}
    \toprule
    \textbf{Method} & \textbf{Runtime (rel.)} & \textbf{Allocations (rel.)} & \textbf{Speedup} \\
    \midrule
    Naïve Julia      & 1.00 & 1.00 & $1.0\times$ \\
    Optimized Julia  & 0.12 & 0.05 & $8.3\times$ \\
    SciML baseline   & 0.10 & 0.04 & $10.0\times$ \\
    \bottomrule
  \end{tabular}
  {\captionsetup{hypcap=false}%
  \captionof{table}{Placeholder benchmark table comparing relative runtime, memory allocations and overall speedup factors.}}
\end{center}

\medskip
\hrulefill
\medskip

% --- END MIDDLE COLUMN
\switchcolumn

% ---- TODO WHAT GOES HERE? --- Conclusion comes last
%------------------------ Conclusion ------------------------
\section*{Conclusion: Future Work for SciML}

\noindent
Symbolic computation can automate derivations and optimizations that would otherwise
be performed manually, improving both performance and reliability.  For example, the 
Rössler system \eqref{eq:rossler} is linear in two unknowns and can be simplified by 
substitution; \texttt{ModelingToolkit.jl} can automate this process and generates 
optimized code for the resulting system.

\medskip

\noindent
By carefully crafting our Julia implementation—using in-place updates, 
static arrays and solver-aware techniques—we achieved C-like performance 
without sacrificing code clarity.  Eliminating unnecessary allocations and 
choosing the right data layout allow simulation code to scale up to large problems,
be reused in future research and be shared confidently with the community.

\medskip

\noindent
Choosing the right integrator is crucial: an explicit scheme is sufficient for 
the non-stiff Rössler system, but implicit methods are necessary for stiff problems.
The SciML ecosystem provides a rich set of tools for selecting and tuning the best 
solver for your problem; don’t guess—measure with \texttt{BenchmarkTools.jl}!

\medskip

\noindent
%------------------------ REFERENCES ------------------------
\nocite{*}
\bibliographystyle{plain}
\bibliography{mybib}

\end{paracol}
\end{document}