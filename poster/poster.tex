\documentclass[a0,landscape]{a0poster}
\input{preamble.tex} % after the basics
\definecolor{PMS310C}{RGB}{62,192,197} % Official MTU colorway



\begin{document}
\begin{tikzpicture}[remember picture,overlay]
  \node[anchor=north east, xshift=-5mm, yshift=-5mm] 
    at (current page.north east) {\includegraphics[height=7cm]{logo.png}};
\end{tikzpicture}

%----------------------------------------------------------------------------------------
% HEADER
%----------------------------------------------------------------------------------------
\vspace*{-2cm} % reduce magnitude if anything touches the top edge

{\setlength{\tabcolsep}{0pt}%
\renewcommand{\arraystretch}{1.05}%

\noindent
\begin{minipage}[t]{0.34\textwidth}
  {\Huge\bfseries\color{PMS310C} POC to Production}\\[1ex]
  {\Large\itshape Fast Julia Differential Equation Solving}\\[2ex]
  {\large Daniel Henderson}\\
  {\normalsize Michigan Technological University}\\
  {\small\ttfamily dphender@mtu.edu}
\end{minipage}%
\begin{minipage}[t]{0.24\textwidth}\vspace{0pt}
{\Large\textbf{Abstract}}
{\normalsize
We compare naïve and optimized Julia implementations for solving
small to moderate systems of ODEs
$\vx'(t) = \vf(\vx, t) \, : \, \dim(\vx) \leq 10^2$.
We demonstrate how a sequence of targeted changes---in-place updates, type stability, and
solver selection---can yield order-of-magnitude speedups while preserving code clarity and
reproducibility. By applying these techniques, we achieve a 10x speedup over the original
code and demonstrate how to scale up to larger systems.}
\end{minipage}
}

\vspace{0.20cm}
\hrulefill
\vspace{0.20cm}

%----------------------------------------------------------------------------------------
% BODY (3 COLUMNS)
%----------------------------------------------------------------------------------------
% ...existing code...

%----------------------------------------------------------------------------------------
% BODY (3 COLUMNS)
%----------------------------------------------------------------------------------------
\begin{paracol}{3}
  \setcolumnwidth{0.28\textwidth,0.44\textwidth,0.28\textwidth}


%------------------------ INTRODUCTION ------------------------
\section*{Introduction}

\noindent
Numerical solution of differential equations underpins many scientific and
engineering workflows. Modern simulation pipelines often require millions of
right-hand side evaluations over long horizons or large parameter sweeps, so
even modest inefficiencies can waste substantial time and memory.

\medskip

\noindent
We demonstrate how to optimize a simple ODE solver pipeline from prototype to
production using Julia, a high-level, high-performance language for numerical
computation \cite{Julia-2017}. Our first case study focuses on the chaotic
system,
\begin{flalign*}
  \tag{Rössler} \label{eq:rossler}
  \quad \vx'(t) = \vf(\vx,\, \vp, \, t) = 
  \vect{
    -x_2 \,-\, x_3 \\
    x_1 \,+\, a x_2 \\
    b \,+\, x_3 (x_1 \,-\, c) \\
  }
    ~ : ~ \begin{cases*}
    \vx = (x_1, \, x_2, \, x_3)^\top ~(\text{state variables})\\
    \vp = (a, \, b, \, c)^\top ~(\text{parameters}) \\
  \end{cases*}, &&
\end{flalign*}
introduced as a simplification of the Lorenz attractor
by Otto Rössler in 1976 \cite{Rössler-1976}.

\medskip

\noindent
We seek $\vx(t)$ for all $t \in [t_0, t_f]$ subject to $\vx(t_0)=\vx_0$, with fixed
parameters $\vp$. ``Production`` performance is achieved using \texttt{DifferentialEquations.jl} (SciML)
\cite{DifferentialEquations.jl-2017}, building on the SciML techniques
\cite{SciMLTutorial} for optimizing the model code $\vf$.

\medskip

\noindent
\textbf{Case Study 1.} Compares out-of-place vs.\ in-place Julia implementations
of the Rössler system (\ref{eq:rossler}), and a type-stable fixed-size variant:
\begin{itemize}
  \item naïve out-of-place code allocates a new array each RHS call, triggering garbage collection;
  \item optimized in-place code eliminates per-call allocations for \texttt{Vector}-based states; and
  \item a \texttt{StaticArrays.jl} / \texttt{SVector} implementation enables stack allocation, often fastest for very small systems.
\end{itemize}

\noindent
\textbf{Case Study 2.} Performs parameter sweep over $\vp$ requires solving many independent ODE instances.
We use Julia broadcasting and array programming to reduce overhead:
\begin{itemize}
  \item fused broadcasts, BLAS operations, and matrix-free stencils to avoid per-step allocations; and
  \item solver choice (explicit Runge--Kutta vs.\ implicit BDF methods) to further improve performance for stiff problems.
\end{itemize}

\noindent
\textbf{Future of SciML is Symbolic--Numeric.} Discuss how symbolic manipulation
simplifies (\ref{eq:rossler}) via substitution, highlighting structure (two linear equations in $\vx$ and one quadratic).

\medskip
\hrulefill
\medskip

% --- Figure placeholder: overview / pipeline ---
\medskip
\begin{center}
  \includegraphics[width=0.4\linewidth]{tb-pipeline.png}
  \captionof{figure}{%
    Lorem ipsum dolor sit amet, consectetur adipiscing elit. 
    Placeholder overview diagram of the proof-of-concept to 
    production ODE solver workflow.}
\end{center}

% --- END FIRST COLUMN
\switchcolumn


% ------------------------ Ideally: Start of Middle Column ------------------------
\section*{Case Study 1: Optimizing the Rössler ODE System}

\noindent
\begin{minipage}{\linewidth}
\centering
\begin{minipage}[t]{0.475\linewidth} % Left: out-of-place
\begin{lstlisting}[
  language=Julia,
  linewidth=\linewidth,
  breaklines=true,
  caption={``Out-of-place`` allocates a new heap array each call.},
  label={lst:naive-out-of-place}
]
function rossler(vx, vp, t)
    dx1 = -vx[2] - vx[3]
    dx2 =  vx[1] + vp[1] * vx[2]
    dx3 =  vp[2] + vx[3] * (vx[1] - vp[3])
    return [dx1, dx2, dx3]
end
\end{lstlisting}
\td{\small \texttt{@allocated rossler([1.0,1.0,1.0], (0.1,0.1,12), 0.0)}
  reports 240 bytes per call (typical).}
\end{minipage}
\hfill
\begin{minipage}[t]{0.475\linewidth} % Right: in-place
\begin{lstlisting}[
  language=Julia,
  linewidth=\linewidth,
  breaklines=true,
  caption={``In-place`` ODE system, utilizing cache array (`vx` passed-by-reference)
  for state values. Avoids per-call allocations and typically fastest for `Vector`-based states.},
  label={lst:in-place}
]
function rossler!(dx, vx, vp, t)
    x1 = vx[1]
    dx[1] = -vx[2] - vx[3]
    dx[2] =  x1 + vp[1] * vx[2]
    dx[3] =  vp[2] + vx[3] * (x1 - vp[3])
    return nothing
end
\end{lstlisting}
\end{minipage}
\vspace{0.75em}
\begin{lstlisting}[
  language=Julia,
  linewidth=\linewidth,
  breaklines=true,
  caption={Fixed-size, stack-allocated \texttt{\cite{StaticArrays.jl-2017}} version.},
  label={lst:static}
]
function rossler_static(vx, vp, t)
    x1 = vx[1]
    dx1 = -vx[2] - vx[3]
    dx2 =  x1 + vp[1] * vx[2]
    dx3 =  vp[2] + vx[3] * (x1 - vp[3])
    return @SVector [dx1, dx2, dx3]
end
\end{lstlisting}
\end{minipage}

\noindent
\subsection*{Results/Analysis}

% --- Figure placeholder: speedup plot ---'
\noindent
\begin{minipage}{\linewidth}
\centering
\begin{minipage}[t]{0.475\linewidth} % Left: out-of-place
  \includegraphics[width=0.95\linewidth]{rk4_time_normalized.png}
  \captionof{figure}{%
    Lorem ipsum dolor sit amet, consectetur adipiscing elit. 
    Placeholder benchmark plot comparing naïve and optimized 
    Julia implementations for a Rössler-type ODE system.}
\end{minipage}
\hfill
\begin{minipage}[t]{0.475\linewidth} % Right: in-place
  \includegraphics[width=0.95\linewidth]{rk4_time_normalized.png}
  \captionof{figure}{%
    Lorem ipsum dolor sit amet, consectetur adipiscing elit. 
    Placeholder benchmark plot comparing naïve and optimized 
    Julia implementations for a Rössler-type ODE system.}
\end{minipage}
\end{minipage}
% --- Table placeholder: benchmark numbers ---
\begin{center}
  \begin{tabular}{lccc}
    \toprule
    \textbf{Method} & \textbf{Runtime (rel.)} & \textbf{Allocations (rel.)} & \textbf{Speedup} \\
    \midrule
    Naïve Julia      & 1.00 & 1.00 & $1.0\times$ \\
    Optimized Julia  & 0.12 & 0.05 & $8.3\times$ \\
    SciML baseline   & 0.10 & 0.04 & $10.0\times$ \\
    \bottomrule
  \end{tabular}
  \captionof{table}{%
    Lorem ipsum dolor sit amet, consectetur adipiscing elit. 
    Placeholder benchmark table comparing relative runtime, 
    memory allocations, and overall speedup factors.}
\end{center}

\medskip
\hrulefill
\medskip

\noindent
\section*{Case Study 2: Parameter Sweep and Solver Selection}
\td{TODO: Structurally, Case 2 and Case 1 are logically seperate but mirror one another, here
we just consider a larger system and a parameter sweep, but the same principles apply.}

% --- END MIDDLE COLUMN
\switchcolumn

%------------------------ Conclusion ------------------------
\section*{Conclusion: Key Takeaways for Production-Quality Code}

\noindent
Symbolic computation can automate derivations and optimizations that 
we’d otherwise do by hand, improving both performance and reliability.
For example, the Rössler system (\ref{eq:rossler}) is linear in two of its
unknowns, and can be simplified by substitution. \texttt{ModelingToolkit.jl} can
automate this process, and even generate optimized code for the resulting system.

\medskip

\noindent
Pellentesque habitant morbi tristique senectus et netus et malesuada fames ac
turpis egestas. Sed posuere consectetur est at lobortis. Vestibulum id ligula
porta felis euismod semper.

%------------------------ Conclusion ------------------------
\section*{Conclusion: Key Takeaways for Production-Quality Code}

\noindent
By carefully crafting our Julia implementation of the Rössler system – using in-place updates,
static arrays, and solver-aware techniques – we achieved C-like performance without sacrificing 
code clarity. Performance through Non-Allocation and developer productivity go hand-in-hand in Julia!
By following these practices, your simulation code can scale up to large problems, be reused in 
future research, and even shared confidently with the community as reliable software.

\medskip
\noindent
One must also remeber the importance of choosing the right integrator, here an explicit
scheme is sufficient for the non-stiff Rössler system, but implicit methods are
necessary for stiff problems. The SciML ecosystem provides a rich set of tools for
selecting and tuning the best solver for your problem; don't guess, measure
with \texttt{BenchmarkTools.jl}!

\medskip

\noindent
%------------------------ REFERENCES ------------------------
\nocite{*}
\bibliographystyle{plain}
\bibliography{mybib}

\end{paracol}
\end{document}
